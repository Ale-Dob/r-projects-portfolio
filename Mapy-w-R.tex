% Options for packages loaded elsewhere
\PassOptionsToPackage{unicode}{hyperref}
\PassOptionsToPackage{hyphens}{url}
\documentclass[
]{article}
\usepackage{xcolor}
\usepackage[margin=1in]{geometry}
\usepackage{amsmath,amssymb}
\setcounter{secnumdepth}{-\maxdimen} % remove section numbering
\usepackage{iftex}
\ifPDFTeX
  \usepackage[T1]{fontenc}
  \usepackage[utf8]{inputenc}
  \usepackage{textcomp} % provide euro and other symbols
\else % if luatex or xetex
  \usepackage{unicode-math} % this also loads fontspec
  \defaultfontfeatures{Scale=MatchLowercase}
  \defaultfontfeatures[\rmfamily]{Ligatures=TeX,Scale=1}
\fi
\usepackage{lmodern}
\ifPDFTeX\else
  % xetex/luatex font selection
\fi
% Use upquote if available, for straight quotes in verbatim environments
\IfFileExists{upquote.sty}{\usepackage{upquote}}{}
\IfFileExists{microtype.sty}{% use microtype if available
  \usepackage[]{microtype}
  \UseMicrotypeSet[protrusion]{basicmath} % disable protrusion for tt fonts
}{}
\makeatletter
\@ifundefined{KOMAClassName}{% if non-KOMA class
  \IfFileExists{parskip.sty}{%
    \usepackage{parskip}
  }{% else
    \setlength{\parindent}{0pt}
    \setlength{\parskip}{6pt plus 2pt minus 1pt}}
}{% if KOMA class
  \KOMAoptions{parskip=half}}
\makeatother
\usepackage{color}
\usepackage{fancyvrb}
\newcommand{\VerbBar}{|}
\newcommand{\VERB}{\Verb[commandchars=\\\{\}]}
\DefineVerbatimEnvironment{Highlighting}{Verbatim}{commandchars=\\\{\}}
% Add ',fontsize=\small' for more characters per line
\usepackage{framed}
\definecolor{shadecolor}{RGB}{248,248,248}
\newenvironment{Shaded}{\begin{snugshade}}{\end{snugshade}}
\newcommand{\AlertTok}[1]{\textcolor[rgb]{0.94,0.16,0.16}{#1}}
\newcommand{\AnnotationTok}[1]{\textcolor[rgb]{0.56,0.35,0.01}{\textbf{\textit{#1}}}}
\newcommand{\AttributeTok}[1]{\textcolor[rgb]{0.13,0.29,0.53}{#1}}
\newcommand{\BaseNTok}[1]{\textcolor[rgb]{0.00,0.00,0.81}{#1}}
\newcommand{\BuiltInTok}[1]{#1}
\newcommand{\CharTok}[1]{\textcolor[rgb]{0.31,0.60,0.02}{#1}}
\newcommand{\CommentTok}[1]{\textcolor[rgb]{0.56,0.35,0.01}{\textit{#1}}}
\newcommand{\CommentVarTok}[1]{\textcolor[rgb]{0.56,0.35,0.01}{\textbf{\textit{#1}}}}
\newcommand{\ConstantTok}[1]{\textcolor[rgb]{0.56,0.35,0.01}{#1}}
\newcommand{\ControlFlowTok}[1]{\textcolor[rgb]{0.13,0.29,0.53}{\textbf{#1}}}
\newcommand{\DataTypeTok}[1]{\textcolor[rgb]{0.13,0.29,0.53}{#1}}
\newcommand{\DecValTok}[1]{\textcolor[rgb]{0.00,0.00,0.81}{#1}}
\newcommand{\DocumentationTok}[1]{\textcolor[rgb]{0.56,0.35,0.01}{\textbf{\textit{#1}}}}
\newcommand{\ErrorTok}[1]{\textcolor[rgb]{0.64,0.00,0.00}{\textbf{#1}}}
\newcommand{\ExtensionTok}[1]{#1}
\newcommand{\FloatTok}[1]{\textcolor[rgb]{0.00,0.00,0.81}{#1}}
\newcommand{\FunctionTok}[1]{\textcolor[rgb]{0.13,0.29,0.53}{\textbf{#1}}}
\newcommand{\ImportTok}[1]{#1}
\newcommand{\InformationTok}[1]{\textcolor[rgb]{0.56,0.35,0.01}{\textbf{\textit{#1}}}}
\newcommand{\KeywordTok}[1]{\textcolor[rgb]{0.13,0.29,0.53}{\textbf{#1}}}
\newcommand{\NormalTok}[1]{#1}
\newcommand{\OperatorTok}[1]{\textcolor[rgb]{0.81,0.36,0.00}{\textbf{#1}}}
\newcommand{\OtherTok}[1]{\textcolor[rgb]{0.56,0.35,0.01}{#1}}
\newcommand{\PreprocessorTok}[1]{\textcolor[rgb]{0.56,0.35,0.01}{\textit{#1}}}
\newcommand{\RegionMarkerTok}[1]{#1}
\newcommand{\SpecialCharTok}[1]{\textcolor[rgb]{0.81,0.36,0.00}{\textbf{#1}}}
\newcommand{\SpecialStringTok}[1]{\textcolor[rgb]{0.31,0.60,0.02}{#1}}
\newcommand{\StringTok}[1]{\textcolor[rgb]{0.31,0.60,0.02}{#1}}
\newcommand{\VariableTok}[1]{\textcolor[rgb]{0.00,0.00,0.00}{#1}}
\newcommand{\VerbatimStringTok}[1]{\textcolor[rgb]{0.31,0.60,0.02}{#1}}
\newcommand{\WarningTok}[1]{\textcolor[rgb]{0.56,0.35,0.01}{\textbf{\textit{#1}}}}
\usepackage{graphicx}
\makeatletter
\newsavebox\pandoc@box
\newcommand*\pandocbounded[1]{% scales image to fit in text height/width
  \sbox\pandoc@box{#1}%
  \Gscale@div\@tempa{\textheight}{\dimexpr\ht\pandoc@box+\dp\pandoc@box\relax}%
  \Gscale@div\@tempb{\linewidth}{\wd\pandoc@box}%
  \ifdim\@tempb\p@<\@tempa\p@\let\@tempa\@tempb\fi% select the smaller of both
  \ifdim\@tempa\p@<\p@\scalebox{\@tempa}{\usebox\pandoc@box}%
  \else\usebox{\pandoc@box}%
  \fi%
}
% Set default figure placement to htbp
\def\fps@figure{htbp}
\makeatother
\setlength{\emergencystretch}{3em} % prevent overfull lines
\providecommand{\tightlist}{%
  \setlength{\itemsep}{0pt}\setlength{\parskip}{0pt}}
\usepackage{bookmark}
\IfFileExists{xurl.sty}{\usepackage{xurl}}{} % add URL line breaks if available
\urlstyle{same}
\hypersetup{
  pdftitle={Mapy w R},
  hidelinks,
  pdfcreator={LaTeX via pandoc}}

\title{Mapy w R}
\author{}
\date{\vspace{-2.5em}2025-12-09}

\begin{document}
\maketitle

\begin{Shaded}
\begin{Highlighting}[]
\FunctionTok{plot}\NormalTok{(}\FunctionTok{st\_geometry}\NormalTok{(mapa.gdansk))}
\end{Highlighting}
\end{Shaded}

\pandocbounded{\includegraphics[keepaspectratio]{Mapy-w-R_files/figure-latex/unnamed-chunk-1-1.pdf}}

\begin{Shaded}
\begin{Highlighting}[]
\CommentTok{\#z dodatkami:}
\FunctionTok{ggplot}\NormalTok{(}\AttributeTok{data =}\NormalTok{ mapa.gdansk)}\SpecialCharTok{+}
  \FunctionTok{geom\_sf}\NormalTok{(}\AttributeTok{fill =} \StringTok{"orange"}\NormalTok{,}
          \AttributeTok{color=} \StringTok{"white"}\NormalTok{)}\SpecialCharTok{+}
  \FunctionTok{theme\_void}\NormalTok{()}
\end{Highlighting}
\end{Shaded}

\pandocbounded{\includegraphics[keepaspectratio]{Mapy-w-R_files/figure-latex/unnamed-chunk-2-1.pdf}}

\subsection{Uproszczenie granic}\label{uproszczenie-granic}

\pandocbounded{\includegraphics[keepaspectratio]{Mapy-w-R_files/figure-latex/pressure-1.pdf}}

\subsection{Kartogram właściwy prosty (skokowy), przedstawiający gęstość
zaludnienia}\label{kartogram-wux142aux15bciwy-prosty-skokowy-przedstawiajux105cy-gux119stoux15bux107-zaludnienia}

\begin{Shaded}
\begin{Highlighting}[]
\NormalTok{mapa.gda}\SpecialCharTok{$}\NormalTok{LICZBA\_MIE[mapa.gda}\SpecialCharTok{$}\NormalTok{LICZBA\_MIE}\SpecialCharTok{==}\DecValTok{0}\NormalTok{]}\OtherTok{\textless{}{-}} \ConstantTok{NA}

\CommentTok{\#nadpisujemy dane:}
\NormalTok{mapa.gda}\SpecialCharTok{$}\NormalTok{GESTOSC }\OtherTok{\textless{}{-}}\NormalTok{ mapa.gda}\SpecialCharTok{$}\NormalTok{LICZBA\_MIE}\SpecialCharTok{/}\NormalTok{mapa.gda}\SpecialCharTok{$}\NormalTok{POWIERZCHN}

\FunctionTok{stripchart}\NormalTok{(mapa.gda}\SpecialCharTok{$}\NormalTok{GESTOSC, }\AttributeTok{jitter =} \StringTok{"jitter"}\NormalTok{) }\CommentTok{\#podglądamy dane, żeby dostosowac skalę}
\FunctionTok{grid}\NormalTok{()}
\end{Highlighting}
\end{Shaded}

\pandocbounded{\includegraphics[keepaspectratio]{Mapy-w-R_files/figure-latex/unnamed-chunk-3-1.pdf}}

\begin{Shaded}
\begin{Highlighting}[]
\FunctionTok{library}\NormalTok{(RColorBrewer)}

\FunctionTok{ggplot}\NormalTok{(}\AttributeTok{data =}\NormalTok{ mapa.gda)}\SpecialCharTok{+}
  \FunctionTok{geom\_sf}\NormalTok{(}\FunctionTok{aes}\NormalTok{(}\AttributeTok{fill=}\NormalTok{GESTOSC))}\SpecialCharTok{+}
  \FunctionTok{theme\_void}\NormalTok{()}\SpecialCharTok{+}
  \FunctionTok{scale\_fill\_fermenter}\NormalTok{(}
    \AttributeTok{name =} \StringTok{"Gęstość}\SpecialCharTok{\textbackslash{}n}\StringTok{zaludnienia"}\NormalTok{,}
    \AttributeTok{guide =} \FunctionTok{guide\_legend}\NormalTok{(),}
    \AttributeTok{palette =} \StringTok{"PuBuGn"}\NormalTok{,}
    \AttributeTok{direction =} \DecValTok{1}\NormalTok{,}
    \AttributeTok{na.value =} \StringTok{"grey90"}\NormalTok{,}
    \AttributeTok{breaks =} \FunctionTok{c}\NormalTok{(}\DecValTok{500}\NormalTok{,}\DecValTok{1000}\NormalTok{,}\DecValTok{2000}\NormalTok{,}\DecValTok{4000}\NormalTok{,}\DecValTok{6000}\NormalTok{,}\DecValTok{8000}\NormalTok{)}
\NormalTok{  )}
\end{Highlighting}
\end{Shaded}

\pandocbounded{\includegraphics[keepaspectratio]{Mapy-w-R_files/figure-latex/unnamed-chunk-3-2.pdf}}
\#\# Kartodiagram, przedstawiający liczbę ludności

\begin{Shaded}
\begin{Highlighting}[]
\NormalTok{mapa.gda}\SpecialCharTok{$}\NormalTok{srodki }\OtherTok{\textless{}{-}} \FunctionTok{st\_centroid}\NormalTok{(mapa.gda}\SpecialCharTok{$}\NormalTok{geometry)}

\FunctionTok{ggplot}\NormalTok{(}\AttributeTok{data=}\NormalTok{mapa.gda)}\SpecialCharTok{+}
  \FunctionTok{geom\_sf}\NormalTok{()}\SpecialCharTok{+}
  \FunctionTok{geom\_sf}\NormalTok{(}\FunctionTok{aes}\NormalTok{(}\AttributeTok{geometry=}\NormalTok{srodki,}
              \AttributeTok{size =}\NormalTok{ LICZBA\_MIE),}
          \AttributeTok{fill=} \StringTok{"deepskyblue"}\NormalTok{,}
          \AttributeTok{alpha =} \FloatTok{0.5}\NormalTok{,}
          \AttributeTok{shape =} \DecValTok{22}\NormalTok{)}\SpecialCharTok{+}
  \FunctionTok{theme\_void}\NormalTok{()}\SpecialCharTok{+}
  \FunctionTok{scale\_size\_continuous}\NormalTok{(}
    \AttributeTok{name =} \StringTok{"liczba mieszkańców"}\NormalTok{,}
    \AttributeTok{range =} \FunctionTok{c}\NormalTok{(}\DecValTok{2}\NormalTok{,}\DecValTok{12}\NormalTok{), }\CommentTok{\#zakres rozmiaru punktów}
    \AttributeTok{breaks =} \FunctionTok{c}\NormalTok{(}\DecValTok{2000}\NormalTok{, }\DecValTok{5000}\NormalTok{, }\DecValTok{10000}\NormalTok{, }\DecValTok{20000}\NormalTok{),}
    \AttributeTok{guide =} \FunctionTok{guide\_legend}\NormalTok{(}\AttributeTok{reverse =}\NormalTok{ T)}
\NormalTok{  )}
\end{Highlighting}
\end{Shaded}

\begin{verbatim}
## Warning: Removed 2 rows containing missing values or values outside the scale range
## (`geom_sf()`).
\end{verbatim}

\pandocbounded{\includegraphics[keepaspectratio]{Mapy-w-R_files/figure-latex/unnamed-chunk-4-1.pdf}}

\end{document}
